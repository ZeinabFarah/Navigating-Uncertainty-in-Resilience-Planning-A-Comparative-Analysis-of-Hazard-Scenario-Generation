%%%%%%%%%%%%%%%%%%%%%%%%%%%%%%%%%%%%%%%%%%%%%%%%%%%%%%%%%%%%%%%%%%%
%   LaTeX source code to approximate a NIST Technical report
%	DOI and CrossMark watermark will be added on final PDF
% 	Developed by K. Miller, kmm5@nist.gov 
%	Last updated: 21-June-2022
%%%%%%%%%%%%%%%%%%%%%%%%%%%%%%%%%%%%%%%%%%%%%%%%%%%%%%%%%%%%%%%%%%%

%%%%%%%%%%%%%%%%%%%%%%%%%%%%%%%%%%%%%%%%%%%%%%%%%%%%%%%%%%%%%%%%%%%
% Document class and template package DO NOT DELETE
%%%%%%%%%%%%%%%%%%%%%%%%%%%%%%%%%%%%%%%%%%%%%%%%%%%%%%%%%%%%%%%%%%%
\documentclass[12pt]{article}
\usepackage{Style/settings}

%%%%%%%%%%%%%%%%%%%%%%%%%%%%%%%%%%%%%%%%%%%%%%%%%%%%%%%%%%%%%%%%%%%%%%%
% DRAFT: TRUE OR FALSE?
% REQUIRED!!! Select either Draft (toggletrue) or Final (togglefalse)
%%%%%%%%%%%%%%%%%%%%%%%%%%%%%%%%%%%%%%%%%%%%%%%%%%%%%%%%%%%%%%%%%%%%%%%
\newtoggle{draft}
% \toggletrue{draft} % use if this is a Draft
\togglefalse{draft} % use if this is Final

\iftoggle{draft}{
    \usepackage{lineno}
    \linenumbers
}

%%%%%%%%%%%%%%%%%%%%%%%%%%%%%%%%%%%%%%%%%%%%%%%%%%%%%%%%%%%%%%%%%%%%%%%
% Publication Metadata
%%%%%%%%%%%%%%%%%%%%%%%%%%%%%%%%%%%%%%%%%%%%%%%%%%%%%%%%%%%%%%%%%%%%%%%
\newcommand{\pubseries}{Series [Technical Note]} % Replace with a series title from the list below %
% {Advanced Manufacturing Series}
% {Data Collection Instruments}
% {Economic Analysis Brief}
% {Grant/Contractor Report}
% {Handbook}
% {IR}
% {Special Publication}
% {Technical Note}
% {Technology Transfer Brief}
% {NCSTAR}
\newcommand{\pubnumber}{Publication Identifier} 
\newcommand{\DOI}{https://doi.org/10.6028/NIST.XXX.XXXX}
\newcommand{\pubmonth}{Month}
\newcommand{\pubyear}{Year}
\newcommand{\erratadate}{MM-DD-YYYY}
\newcommand{\pubtitle}{Navigating Uncertainty in Resilience Planning: A Comparative Analysis of Hazard Scenario Generation Methods}
% \newcommand{\pubsubtitle}{Subtitle} % delete if no subtitle %
% \newcommand{\draftstage}{Draft Stage} % Draft Stage is REQUIRED for drafts/preprints; Draft stages are listed in the NIST Publication Identifier guidance: https://www.nist.gov/nist-research-library/nist-technical-series-publications-author-instructions#pubid   %
\newcommand{\authorlist}{Zeinab Farahmandfar, Kenneth Harrison}
\newcommand{\authorone}{Zeinab Farahmandfar}
\newcommand{\authortwo}{Kenneth Harrison}

\newcommand{\pubabstract}{This paper presents a comparative analysis of the performance of Importance Sampling (IS) and Crude Monte Carlo (MC) simulation methods for generating a representative hazard scenario, with a particular focus on their implications for community resilience. The ability to accurately represent a hazard scenario is crucial in assessing and enhancing community resilience, and this study sheds light on the critical importance of choosing the right scenario generation approach.\\
Using an illustrative example, we provide a clear and accessible visualization of the performance of both MC and IS methods. To facilitate this comparison, we begin by defining a pre-mitigation hypothetical loss function, serving as a baseline for assessing resilience. Subsequently, three distinct resilience actions are introduced, each resulting in a unique loss function representing post-mitigation losses.\\
By generating hazard scenarios using both MC and IS techniques, we demonstrate that the choice of sampling method significantly influences the decision-making process regarding resilience actions. Our analysis reveals that an inaccurate or inadequate representation of a hazard scenario can lead to suboptimal decisions, potentially jeopardizing community resilience efforts.}
\newcommand{\keywords}{Uncertainty; Monte Carlo Simulation; Importance Sampling; Markov Chain Monte Carlo; Community Resilience; Disaster; Natural Hazards; Hazard Scenario.}
\newcommand{\publang}{English} %If translation, change to correct language%

\usepackage{Style/pdfproperties}

\newcommand{\SampleSize}{20}
\newcommand{\TrueValue}{0.079}
\newcommand{\MCResult}{0.045}
\newcommand{\TruncatedMCResult}{0.646}
\newcommand{\TraditionalISResult}{0.068}
\newcommand{\MCMCISResult}{0.079}

\newcommand{\MCError}{0.883}
\newcommand{\TruncatedMCError}{7.217}
\newcommand{\TraditionalISError}{0.755}
\newcommand{\MCMCISError}{0.036}
