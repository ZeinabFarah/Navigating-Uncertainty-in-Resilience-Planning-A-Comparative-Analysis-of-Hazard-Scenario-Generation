\section{Introduction}
    In recent years, the world has seen a worrying increase in the number and severity of disasters caused by natural hazards. In 2022, the destructive Hurricane Ian caused  \$112.9 billion in damages, and the relentless Western and Central Drought/Heat Wave, resulted in \$22.1 billion in economic losses \cite{noaa_billion-dollar_2023}. The year 2023 has brought an even more troubling reality—a record-breaking 23 confirmed weather and climate disaster events, each with losses exceeding \$1 billion \cite{noaa_assessing_2023}.
    
    In the face of such daunting challenges, the urgent need to make communities more resilient has become more crucial than ever. A critical aspect of this effort is the accurate representation of hazard scenarios, which serves as the foundation for informed decision-making aimed at safeguarding lives, infrastructure, and the environment. The necessity to accurately depict hazard scenarios for informed decision-making is an urgent and practical requirement. 
    
    Considering this background, this paper investigates hazard scenario generation, focusing on two fundamental simulation methods: Monte Carlo (MC) simulation and the Importance Sampling (IS) technique. Our primary objective is twofold. First, we illustrate the performance of the MC simulation and IS technique as hazard scenario generation approaches in a clear and understandable manner. Second, we will demonstrate the importance of selecting appropriate sampling methods to generate a representative hazard scenario. Creating scenarios can exert a significant influence on the decision-making process in the context of community resilience. We assert that an inadequate or inaccurate representation of hazard scenarios can lead to suboptimal decisions, thus jeopardizing the efficacy of resilience efforts.

    To begin, we establish a baseline by defining a pre-mitigation hypothetical loss function. A loss function serves as a quantitative measure that assesses the adverse impact or damage caused by a hazard on a system, community, or infrastructure. It can encompass various forms of loss, including cost, recovery time, and social and economic impacts and it can be evaluated using resilience assessment tools such as NIST-ARC \cite{harrison_nist_2023,faiz_risk-averse_2024}, IN-CORE, or R2D \cite{mckenna_nheri-simcenterr2dtool_2024}. This measure forms the basis for evaluating and enhancing resilience strategies. Subsequently, we introduce three distinct resilience actions, each yielding a unique loss function representing post-mitigation conditions. Through the generation of various hazard scenarios, we illuminate the nuanced distinctions that these methods introduce, shedding light on their implications for community resilience.
    
    The paper's subsequent sections are organized as follows: Section 2 offers a brief background and reviews relevant literature concerning the application of MC and IS methods in addressing uncertainty within risk and resilience assessment. Section 3 provides a comprehensive explanation of the importance sampling technique, including its mathematical representation, and introduces the specialized adaptive IS method known as Markov Chain Monte Carlo Importance Sampling (MCMC-IS). Section 4 conducts a comparative analysis between the MC simulation method and the IS techniques, utilizing a one-dimensional numerical example for comparison purposes. Section 5 underscores the significance of precise hazard scenario generation, while our concluding remarks are presented in Section 6.