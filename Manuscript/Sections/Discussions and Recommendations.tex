\section{Discussions and Recommendations}
    In this section, we summarize the findings from our comparative analysis of crude MC and IS methods for hazard scenario generation. Our analysis underscores the substantial influence of the chosen sampling method on the accuracy of hazard scenario representation, thereby directly impacting the decision-making process for community resilience strategies. The crude MC approach, while widely used, demonstrated limitations in capturing the intricacies of complex hazards. In contrast, the IS methods, specifically the MCMC-IS technique, exhibited accuracy in representing intricate risk factors. This accuracy is vital in real-world applications where decisions about resilience actions need to be made with a high degree of confidence.\\
    It should be noted that although we recognize the significance of more intricate analyses involving probabilistic assessments, long-term impact evaluations, and stakeholder engagement, our aim was to emphasize the importance of selecting the most accurate representative hazard scenario. For the sake of simplicity and to maintain the focus on showcasing the importance of accurate hazard scenario selection, we did not consider these additional factors in our current analysis.\\
    We recommend that future research focus on refining methods for selecting representative hazard scenarios, considering diverse and dynamic factors, such as climate change, population growth, and technological advancements. A robust scenario selection process ensures that resilience actions are tailored to the specific challenges faced by communities, enhancing the efficacy of mitigation efforts.