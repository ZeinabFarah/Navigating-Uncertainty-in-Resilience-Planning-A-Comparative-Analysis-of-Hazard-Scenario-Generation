\section{Discussions and Recommendations}
    In this section, we summarize the findings from our comparative analysis of crude MC and IS methods for hazard scenario generation. Our analysis underscores the substantial influence of the chosen sampling method on the accuracy of hazard scenario representation, thereby directly impacting the decision-making process for community resilience strategies. The crude MC approach, while widely used, demonstrated limitations in capturing the intricacies of complex hazards. In contrast, the IS method, specifically the MCMC-IS technique, exhibited remarkable accuracy in representing intricate risk factors. This accuracy is vital in real-world applications where decisions about resilience actions need to be made with a high degree of confidence.
    We recommend that future research focus on refining methods for selecting representative hazard scenarios, considering diverse and dynamic factors, such as climate change, population growth, and technological advancements. A robust scenario selection process ensures that resilience actions are tailored to the specific challenges faced by communities, enhancing the efficacy of mitigation efforts.